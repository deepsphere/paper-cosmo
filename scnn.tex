%% 
%% Copyright 2007, 2008, 2009 Elsevier Ltd
%% 
%% This file is part of the 'Elsarticle Bundle'.
%% ---------------------------------------------
%% 
%% It may be distributed under the conditions of the LaTeX Project Public
%% License, either version 1.2 of this license or (at your option) any
%% later version.  The latest version of this license is in
%%    http://www.latex-project.org/lppl.txt
%% and version 1.2 or later is part of all distributions of LaTeX
%% version 1999/12/01 or later.
%% 
%% The list of all files belonging to the 'Elsarticle Bundle' is
%% given in the file `manifest.txt'.
%% 
%% Template article for Elsevier's document class `elsarticle'
%% with harvard style bibliographic references
%% SP 2008/03/01

\documentclass[preprint,12pt,authoryear]{elsarticle}

%% Use the option review to obtain double line spacing
%% \documentclass[authoryear,preprint,review,12pt]{elsarticle}

%% Use the options 1p,twocolumn; 3p; 3p,twocolumn; 5p; or 5p,twocolumn
%% for a journal layout:
%% \documentclass[final,1p,times,authoryear]{elsarticle}
%% \documentclass[final,1p,times,twocolumn,authoryear]{elsarticle}
%% \documentclass[final,3p,times,authoryear]{elsarticle}
%% \documentclass[final,3p,times,twocolumn,authoryear]{elsarticle}
%% \documentclass[final,5p,times,authoryear]{elsarticle}
%% \documentclass[final,5p,times,twocolumn,authoryear]{elsarticle}


\usepackage[utf8]{inputenc}



%% For including figures, graphicx.sty has been loaded in
%% elsarticle.cls. If you prefer to use the old commands
%% please give \usepackage{epsfig}

%% The amssymb package provides various useful mathematical symbols
\usepackage{amssymb}
\usepackage{color}
\usepackage{url}
\usepackage{hyperref}
\usepackage{graphicx,array}
\usepackage{amsmath}

%% The lineno packages adds line numbers. Start line numbering with
%% \begin{linenumbers}, end it with \end{linenumbers}. Or switch it on
%% for the whole article with \linenumbers.
%% \usepackage{lineno}

\newcommand{\nati}[1]{{\color[rgb]{.1,.6,.1}{#1}}}

\newcommand{\todo}[1]{{\color[rgb]{.6,.1,.6}{#1}}}

\newcommand{\assign}[1]{{\color[rgb]{.8,.5,.8}{Assigned: #1 }}}




\journal{Astronomy and Computing}

\begin{document}

\begin{frontmatter}

%% Title, authors and addresses

%% use the tnoteref command within \title for footnotes;
%% use the tnotetext command for theassociated footnote;
%% use the fnref command within \author or \address for footnotes;
%% use the fntext command for theassociated footnote;
%% use the corref command within \author for corresponding author footnotes;
%% use the cortext command for theassociated footnote;
%% use the ead command for the email address,
%% and the form \ead[url] for the home page:
%% \title{Title\tnoteref{label1}}
%% \tnotetext[label1]{}
%% \author{Name\corref{cor1}\fnref{label2}}
%% \ead{email address}
%% \ead[url]{home page}
%% \fntext[label2]{}
%% \cortext[cor1]{}
%% \address{Address\fnref{label3}}
%% \fntext[label3]{}

\title{Efficient spherical Convolutional Neural Networks with Healpix sampling for cosmological applications}

%% use optional labels to link authors explicitly to addresses:
%% \author[label1,label2]{}
%% \address[label1]{}
%% \address[label2]{}

\author{}

\address{}

\begin{abstract}
%% Text of abstract

\end{abstract}

\begin{keyword}
%% keywords here, in the form: keyword \sep keyword

%% PACS codes here, in the form: \PACS code \sep code

%% MSC codes here, in the form: \MSC code \sep code
%% or \MSC[2008] code \sep code (2000 is the default)

\end{keyword}

\end{frontmatter}

%% \linenumbers

%% main text
\section{Introduction}
\label{sec:intro}

\subsection{Motivation}
\assign{Tomek}

a) Cosmology has a lot of spherical data. Our solution is easy to use and is based on the widely used healpix sampling.
b) It is the most efficient spherical convolution

\subsection{Potential applications}
	\assign{Tomek, Nathanael}

\begin{itemize}

	\item Efficient compression de-compression of 360 videos\cite{su2017learning}
	\item Analysis of spherical cosmological data such as the cosmic microwave background \cite{...}, already done in\cite{he2018analysis}.
\end{itemize}


\subsection{Related work}
\assign{Nathanael}
\cite{cohen2017convolutional} \nati{+ add other graph CNN approaches \cite{...}}
\cite{cohen2018spherical}
\cite{boomsma2017spherical}
\cite{defferrard2016convolutional}

\section{Spherical convolution on sphere using a graph}
\assign{Nathanael}

\begin{itemize}
	\item We build a graph using the healpix sampling
	\item Define Fourier transform and show that the harmonics are visually close to the spherical harmonics
	\item Define spherical convolution using the graph Fourier transform and show heat diffusion example
	\item Show the limits of the approach and explain why we cannot have a perfect spherical convolution with this technique
\end{itemize}

\section{Efficient spherical CNN}
\assign{Michael}


\begin{itemize}
	\item We need efficient convolution, hence the Chebysheff trick
	\item Define spherical CNN using graph CNN
\end{itemize}

\section{Experiments}
\assign{Nathanael, Tomek}

\begin{itemize}
	\item With the entire sphere / different PSD (The PSD features + linear SVM is equivalently good)
	\item Without the entire sphere / same PSD (1 sample: the histogram features + kernelized SVM is equivalently good)
	\item ??
\end{itemize}


\section*{Thanks}

%% The Appendices part is started with the command \appendix;
%% appendix sections are then done as normal sections
%% \appendix

%% \section{}
%% \label{}

%% If you have bibdatabase file and want bibtex to generate the
%% bibitems, please use
%%
\section*{Bibliography}
\bibliographystyle{elsarticle-harv} 
\bibliography{biblio}

\end{document}

\endinput
%%
%% End of file `elsarticle-template-harv.tex'.
